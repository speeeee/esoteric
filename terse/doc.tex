\documentclass[12pt]{report}
\title{the \emph{terse} interpreter}
\author{Spencer Sallay}
\usepackage{amsmath}
\usepackage[pdftex,bookmarks,colorlinks]{hyperref}
\begin{document}
  \maketitle
  \tableofcontents
  \newpage
  \chapter{Introduction}
  \emph{terse} is a very terse, stack-based, concatenative language built for iOS systems
  (specifically iOS 7.0+).  Out of the box, the language will also provide graphics
  capability, allowing the user to write very small programs that can manipulate a 
  canvas.  The aim of the project is to provide a programming language that does not 
  require much writing to execute certain things.  Programming on iOS platforms is 
  practically non-existant, though this is probably for the better.  While terse is a 
  programming language itself, its use is not in writing practical applications, but 
  instead in writing quick programs.

  terse is built using Swift 2.x, while the graphics librarry accompanying it is 
  {CoreGraphics?/GLKit?}.  All information regarding the details of terse are explained
  in the documentation below.
  \chapter{Tutorial}
  This document acts as a complete\footnote{This document is \emph{not} yet complete as
  of \date{}} reference and tutorial for \emph{terse}.  While it is highly recommended
  that the user read the tutorial, there is no expectation to read the entire
  reference.  However, at the end of this tutorial, there will be a list of words 
  (function) that are recommended to be read about in the reference.
  \section{Stack languages}
  As mentioned before, \emph{terse} is a stack-based, concatenative language.  Ignoring
  the "concatenative" part for now, a stack-based language is essentially a language
  where all actions performed are done so on an stack structure (also known as a
  \emph{LIFO-queue}).  A stack is a very simple structure in that it has two
  operations: \emph{push} and \emph{pop}.  Pushing something onto the stack will put
  what is to be pushed onto the stack for later use.  Popping something from the
  stack will take the top element of the stack, remove it, and do \emph{something} with
  it.  The something that it does will be explained later.
  \section{terse as a stack language}
  \verb|terse| works like most high level stack-based languages as it uses a
  concatenative style of programming to represent interactions with the stack.
  For example, the expression \verb|1 2 3| represents the pushing of three numbers,
  \verb|1|, \verb|2|, and \verb|3|, to the stack.  In general, any whitespace character
  is used to separate tokens, while any token is represented by a numeric value or
  a symbol.
  \subsection{Calling functions}
  Being able to only push to the stack is a fairly limited feature.  Most stack-based
  languages implement functions as things that pop items from the stack for use and
  push what is returned from the function.  An example of this using the function,
  \verb|+|, is as follows:
  \newline\newline
  \verb|1 2 3 + +|
  \newline\newline
  This example is a direct representation of the infix counterpart: \verb|(2 + 3) + 1|.
  Another thing to note is that because of how each token is read sequentially, it
  is possible to chain functions emulating function composition.  This is one of
  the main advantages of stack-based languages; for example, to represent the
  composition \begin{math}(f\circ g\circ h)(x)\end{math} in a stack-based
  setting as opposed to most traditional function application languages:
  \verb|x h g f| and \verb|f(g(h(x)))|.  This greatly reduces the amount of typing
  necessary, while arguably not compromising the clarity of it.  Of course, this does
  have a change to lead to some ambiguity when reading code.  Consider the expression,
  \verb|1 2 3 a b|.  Assuming that \verb|a| and \verb|b| are functions, the actual
  amount of arguments each function takes is slightly ambiguous.  Because of this,
  the programmer should make sure to clear those ambiguities if readability is a
  necessity (for example, assuming that \verb|a| takes two arguments and returns one,
  and \verb|b| also takes two arguments and returns 0, it could be rewritten as
  \verb|1 2 a 3 b|).
  \newline\newline
  Finally, all functions' inputs and outputs from now on will be represented by
  their \emph{stack signature}.  For example, the stack signature for \verb|+| is
  \verb|( a b -- c )|.
  \subsection{Creating functions and quotes}
  Before covering functions, it should be explained what quotes are.  Quotes are
  terse's (and a few other stack languages, Joy being the first to implement them)
  representation of first-class functions.  First-class functions in other languages
  are functions that act as literals, meaning that they can be used like numbers,
  strings\footnote{This assumes that strings are counted as literals themselves, like
  how a few LISP variants do.}, etc.  This document will not go very in-depth on what
  first-class functions can do, but they are very imporant to understanding the
  language.  The following is an example of a quote being called:
  \newline\newline
  \verb|2 [ 1 + } !|
  \newline\newline
  The actual quote itself is the expression between the \verb|[| and the \verb|}|.
  Within the quote is the expression, \verb|1 +|.  However, this expression is not
  yet called, and is instead pushed to the stack.  The function, \verb|!|, calls
  the quote on top of the stack, which means that whatever is in the expression is
  essentially pushed directly to the stack, making the expression \verb|2 1 +|.
  This is then evaluated to \verb|3|, as it would normally.  Quotes, as mentioned
  before, can be used like anything else on the stack, meaning it can be subject
  to things like conditional expressions, giving it a lot of potential use.
  \newline\newline
  This relates to functions as all functions are in terse are essentially just named
  quotes.  The following is an example of a function that takes the average of two
  numbers, \verb|avg|:
  \newline\newline
  \verb|: avg + 2 / }|
  \newline\newline
  The first token read, \verb|:|, is special in that it changes the \emph{mode} of
  the reading.  It actually does the same thing as \verb|[| does, as \verb|[| is
  technically another mode-changing function itself.  Because of this, \verb|}|
  acts as the closing of that mode.  However, instead of pushing this to the stack,
  it instead stores it in a list of other functions, where the name of the function
  is \verb|avg| and its body is \verb|2 + /|.  One thing to note is the absence
  of any argument declaration.  Instead, all that are there are the name and body
  of the function.  The reason for this is because the actual stack signature of
  the function can be derived from the body fairly easily.  For example, the
  stack signature of this is \verb|( a b -- c )|.  There is only one time, however,
  where it is impossible to derive the signature, and that is for recursive functions.
  Since they call themselves within themselves, it is impossible to tell what
  the actual signature is.  This is not an issue by default in terse though, as
  the interpreter does not actually bother coming up with the stack signature.
  \newline\newline
  Now that the function is defined, it can now be called.  Calling functions is fairly
  straightfoward, as it only requires that the amount of arguments necessary be on
  the stack and that the function itself is pushed: 
  \verb|avg 1 3| $\leftarrow$ \verb|2|.
\end{document}
